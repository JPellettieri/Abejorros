% Options for packages loaded elsewhere
\PassOptionsToPackage{unicode}{hyperref}
\PassOptionsToPackage{hyphens}{url}
%
\documentclass[
  11pt,
]{article}
\usepackage{amsmath,amssymb}
\usepackage{iftex}
\ifPDFTeX
  \usepackage[T1]{fontenc}
  \usepackage[utf8]{inputenc}
  \usepackage{textcomp} % provide euro and other symbols
\else % if luatex or xetex
  \usepackage{unicode-math} % this also loads fontspec
  \defaultfontfeatures{Scale=MatchLowercase}
  \defaultfontfeatures[\rmfamily]{Ligatures=TeX,Scale=1}
\fi
\usepackage{lmodern}
\ifPDFTeX\else
  % xetex/luatex font selection
\fi
% Use upquote if available, for straight quotes in verbatim environments
\IfFileExists{upquote.sty}{\usepackage{upquote}}{}
\IfFileExists{microtype.sty}{% use microtype if available
  \usepackage[]{microtype}
  \UseMicrotypeSet[protrusion]{basicmath} % disable protrusion for tt fonts
}{}
\makeatletter
\@ifundefined{KOMAClassName}{% if non-KOMA class
  \IfFileExists{parskip.sty}{%
    \usepackage{parskip}
  }{% else
    \setlength{\parindent}{0pt}
    \setlength{\parskip}{6pt plus 2pt minus 1pt}}
}{% if KOMA class
  \KOMAoptions{parskip=half}}
\makeatother
\usepackage{xcolor}
\usepackage[margin=1in]{geometry}
\usepackage{graphicx}
\makeatletter
\def\maxwidth{\ifdim\Gin@nat@width>\linewidth\linewidth\else\Gin@nat@width\fi}
\def\maxheight{\ifdim\Gin@nat@height>\textheight\textheight\else\Gin@nat@height\fi}
\makeatother
% Scale images if necessary, so that they will not overflow the page
% margins by default, and it is still possible to overwrite the defaults
% using explicit options in \includegraphics[width, height, ...]{}
\setkeys{Gin}{width=\maxwidth,height=\maxheight,keepaspectratio}
% Set default figure placement to htbp
\makeatletter
\def\fps@figure{htbp}
\makeatother
\setlength{\emergencystretch}{3em} % prevent overfull lines
\providecommand{\tightlist}{%
  \setlength{\itemsep}{0pt}\setlength{\parskip}{0pt}}
\setcounter{secnumdepth}{5}
\ifLuaTeX
  \usepackage{selnolig}  % disable illegal ligatures
\fi
\usepackage{bookmark}
\IfFileExists{xurl.sty}{\usepackage{xurl}}{} % add URL line breaks if available
\urlstyle{same}
\hypersetup{
  pdftitle={TP final: Efecto de Cafeína y Arginina en la formación de memoria a largo plazo de abejorros},
  pdfauthor={Pellettieri Julieta, Morazzo Nunzi Valentino, Lachavanne Moira, Bollini Guillermina y Maldonado Lucio},
  hidelinks,
  pdfcreator={LaTeX via pandoc}}

\title{TP final: Efecto de Cafeína y Arginina en la formación de memoria
a largo plazo de abejorros}
\author{Pellettieri Julieta, Morazzo Nunzi Valentino, Lachavanne Moira,
Bollini Guillermina y Maldonado Lucio}
\date{15/10/2024}

\begin{document}
\maketitle

\subsection{Introducción:}\label{introducciuxf3n}

En la actualidad el incremento en la producción agropecuaria mediante
prácticas sustentables se volvió una prioridad. La polinización dirigida
es una estrategia en desarrollo que consiste en condicionar
polinizadores como abejas y abejorros para aumentar la producción de
frutales. La mayoría de los estudios para perfeccionar esta técnica se
llevaron a cabo en Apis Mellifera por ser una especie accesible, de uso
generalizado y con un alto grado de domesticación. En los mismos se
encontraron que compuestos presentes naturalmente en concentraciones
trazas en el néctar de las plantas como la cafeína y la arginina
favorecen la formación de memoria a largo término. Bombus Pauloensis es
una especie de abejorro nativa de Argentina empleada para la
polinización dirigida. Sin embargo, debido a su alto costo, difícil
manejo e inaccesibilidad es una especie poco estudiada. Con el objetivo
de mejorar la performance de esta especie en las prácticas de
polinización dirigida se busca evaluar el impacto de la cafeína y la
arginina en la formación de memoria a largo término.

\subsection{Objetivo:}\label{objetivo}

Evaluar si el consumo de cafeína y/o arginina influye en la persistencia
de la memoria a largo plazo. \#\# Hipótesis: Tanto el tratamiento con
cafeína como con arginina, así como la combinación de ambos, tienen un
efecto positivo en la memoria a largo término. El tratamiento combinado
con cafeína y arginina genera un efecto mayor en la memoria a largo
plazo en comparación con el tratamiento de cada compuesto de manera
individual.

\subsection{Descripción del ensayo:}\label{descripciuxf3n-del-ensayo}

En este contexto, a lo largo del año 2024 se llevó a cabo un ensayo
experimental en el Campo Experimental de la Facultad de Ciencias Exactas
y Naturales de la Universidad de Buenos Aires, Argentina, el cual
involucró el uso de seis nidos industriales, de los cuales solo fue
posible evaluar cuatro debido al manejo complejo de la especie. Cada
nido fue evaluado durante el período en que estuvo activo, y luego se
continuaron los ensayos con el siguiente nido. Por lo tanto, en cada día
de ensayo, salvo en los días de transición entre nidos, se evaluaron
individuos pertenecientes al mismo nido.

El ensayo consistió en un entrenamiento asociativo clásico, con seis
exposiciones a Lionanol, un compuesto aromático presente en cientos de
flores, seguido de una recompensa. Esta consistía en una solución
azucarada al 50\% p/p, a la cual se le añadía, según el tratamiento,
cafeína (0,15 mM) (CAF), arginina (0,03 mM) (ARG), ninguno de estos
compuestos (SIN CNA), o ambos (CAF+ARG). También se incluyó un grupo
control, el cual fue recompensado seis veces sin ser expuesto al olor
(SIN OLOR). En cada exposición, el individuo podía elegir tomar la
recompensa o no; solo aquellos individuos que consumieron la recompensa
al menos tres veces fueron considerados entrenados.

A las 24 hs de realizado el entrenamiento se evaluó la respuesta de los
individuos mediante un ensayo PER (Respuesta de Extensión de
Probóscide). El mismo consiste en exponer al abejorro al olor conocido
(en este caso Lionanol) y a un olor novedoso (Nonanal) y evaluar en cada
exposición si el individuo extiende la probóscide a la espera de la
recompensa o no. Solo aquellos individuos que a las 24hs extienden la
probóscide en presencia de Lionanol y no en presencia de Nonanal son
considerados como que recuerdan el olor enseñado en el entrenamiento, a
partir de esta información se definió la variable dicotómica
``Recuerda''.

Tabla 1: Cantidad de abejorros evaluados bajo cada tratamiento en el
período estudiado.

\begin{verbatim}
##      ARG      CAF  CAF+ARG  SIN CNA SIN OLOR 
##       50       48       55       48       28
\end{verbatim}

\subsection{Análisis exploratorio de los
datos.}\label{anuxe1lisis-exploratorio-de-los-datos.}

\subparagraph{Cantidad de ingestas}\label{cantidad-de-ingestas}

Una mayor cantidad de ingestas durante la etapa de entrenamiento se
asocia con una mayor probabilidad de formar memorias a largo plazo.
Generalmente, se considera que con al menos tres ingestas, el individuo
es capaz de formar una asociación entre el estímulo y la recompensa y
generar una memoria a largo término (24 horas). Ni la cafeína ni la
arginina, interfieren en la cantidad de veces que el abejorro toma la
recompensa.

\begin{verbatim}
## `geom_smooth()` using formula = 'y ~ x'
\end{verbatim}

\begin{verbatim}
## Warning in eval(family$initialize): non-integer #successes in a binomial glm!
## Warning in eval(family$initialize): non-integer #successes in a binomial glm!
## Warning in eval(family$initialize): non-integer #successes in a binomial glm!
## Warning in eval(family$initialize): non-integer #successes in a binomial glm!
\end{verbatim}

\includegraphics{Proyecto-final-Biome_files/figure-latex/ingestasTratamiento-1.pdf}

\emph{Figura 1: Porcentaje de respuesta al PER a las 24hs en función de
la cantidad de ingesta y discriminando por tratamiento.}

\begin{verbatim}
##    Estacion         Hora de captura  
##  Length:151         Min.   : 0.3542  
##  Class :character   1st Qu.: 0.4375  
##  Mode  :character   Median : 0.5743  
##                     Mean   : 0.8262  
##                     3rd Qu.: 0.6132  
##                     Max.   :12.0000
\end{verbatim}

\begin{verbatim}
##            Condicionamiento
## Tratamiento  3  4  5  6
##    ARG       1  2  6 34
##    CAF       1  2  5 36
##    CAF+ARG   3  2  3 28
##    SIN CNA   2  2  3 21
##    SIN OLOR  0  0  0  0
\end{verbatim}

\emph{Tabla 2: Cantidad de individuos que se sometieron a cada
tratamiento y que ingirieron distintas veces la recompensa en el
entrenamiento.}

\begin{verbatim}
## # A tibble: 5 x 3
##   Condicionamiento Probabilidad     n
##              <int>        <dbl> <int>
## 1                1          0       1
## 2                3         25       8
## 3                4         22.2     9
## 4                5         52.4    21
## 5                6         42.8   145
\end{verbatim}

\emph{Tabla 3: Porcentaje que respondió al olor condicionado y no al
incondicionado (respuesta al PER) durante la evaluación en función de la
cantidad de ingestas durante el entrenamiento y número de abejorros para
cada uno de estos grupos}

\includegraphics{Proyecto-final-Biome_files/figure-latex/ingestasplot-1.pdf}

\emph{Figura 2: Porcentaje que respondió al PER en función de la
cantidad de ingestas durante el entrenamiento.}

Los datos apoyan el criterio de experto tomado. Además se evidencian 3
posibles casos según la cantidad de ingestas: 1. Cuando el individuo no
toma la recompensa o la toma 1 o 2 veces, no hay registro de respuesta
al PER a las 24 hs. 2. Cuando el individuo toma entre 3 y 4 veces,
aproximadamente el 25\% de los abejorros respondieron al PER a las 24
hs. 3. Cuando el individuo toma más de 4 veces, el 42\% de los abejorros
respondieron al per a las 24 hs.

En función de estos resultados se decidió incluir el número de ingestas
como variable cualitativa ``Cantidad de ingestas (Baja y Alta)'' al
modelo como variable de efectos fijos. Donde Baja incluye a los
individuos que tomaron 3 o 4 veces la recompensa y Alta a los individuos
que tomaron 5 o 6 veces la recompensa.

\subparagraph{Tratamiento}\label{tratamiento}

Se continuó con el análisis exploratorio del efecto del tratamiento en
la retención de la memoria. Por la variabilidad experimental en el
número de ingestas durante el entrenamiento y la utilización de
individuos con al menos tres ingestas los distintos grupos resultaron no
estar balanceados (Anexo 1).

\includegraphics{Proyecto-final-Biome_files/figure-latex/VR-1.pdf}

\emph{Figura 3: Porcentaje de abejorros que respondieron a PER en
función del tratamiento.}

El gráfico muestra el porcentaje de abejorros que respondieron a la
prueba de respuesta (PER) en función de los diferentes tratamientos
(Fig. 4). En general, los tratamientos con arginina (ARG), cafeína (CAF)
y la combinación de ambos (CAF+ARG) muestran una alta tasa de respuesta,
con porcentajes del 50\% para (ARG), 48,7\% para (CAF) y la combinación
de ambos compuestos (ARG+CAF) siendo la que alcanza el valor más alto,
con porcentaje de 52,8\% (Anexo.1). Por otro lado, el tratamiento sin
compuesto no azucarado (SIN CNA) presenta una menor tasa de respuesta.
El control sin olor (SIN OLOR) no muestra ninguna respuesta, por lo
tanto el protocolo aplicado fue adecuado.

\paragraph{Análisis de las variables
aleatorias}\label{anuxe1lisis-de-las-variables-aleatorias}

Como variables aleatorias se incluyeron las variables Nido y Día. Con el
fin de controlar mejor la variabilidad asociada con la estructura
jerárquica de los datos, reduciendo posibles sesgos y mejorando la
robustez de las estimaciones sobre los efectos de los tratamientos
experimentales. La variable Nido captura la alta variabilidad genética,
estacional y comportamental entre nidos. Mientras que la variable Día
refleja fluctuaciones temporales, como cambios ambientales o
relacionados con el experimento. La inclusión de estas variables busca
capturar y controlar fuentes de variabilidad, con el objetivo de evaluar
si mejora la precisión y robustez de las estimaciones del modelo. En
este contexto, es importante considerar que los datos obtenidos del nido
4 fueron recolectados durante el otoño, entre el 25 de marzo y el 17 de
abril. En esta época del año es habitual que la actividad de los nidos
de abejorros decaiga, y que la reina entre en hibernación hasta el
siguiente periodo de floración. Por lo tanto, es razonable suponer que
la actividad de este nido esté alterada y sea menor de lo habitual.
Asimismo, la baja actividad del nido 4 dificultó la toma de datos, lo
que limitó su representación en todos los tratamientos (Tabla 3). En
consecuencia, se anticipa que el nido 4 no seguirá las tendencias
observadas en los demás nidos.

\#\#\#\#\#Nido:

\begin{verbatim}
##    
##     ARG CAF CAF+ARG SIN CNA SIN OLOR
##   2  11  12       8       7        4
##   4   3   0       3       4        3
##   5  21  20      15      14       14
##   6   9   9      11       5        2
##   7   0   1       1       1        0
##   8   1   1       1       1        0
##   9   1   1       0       0        0
\end{verbatim}

\textbf{Tabla 4: Número total de animales en función del tratamiento
para los distintos nidos.}

\includegraphics{Proyecto-final-Biome_files/figure-latex/nidos-1.pdf}

\emph{Figura 4: Gráfico de perfiles de la tasa promedio de aprendizaje
en función del tratamiento para los distintos nidos evaluados.}

En el gráfico se muestran las tasas promedio de aprendizaje en los
distintos tratamientos (ARG, CAF, CAF+ARG y SIN CNA) para los cuatro
nidos (2, 4, 5 y 6). Las tendencias observadas para los nidos 2, 5 y 6
fueron similares entre sí, mostrando valores de tasa de aprendizaje
elevadas para los tratamientos con CNA y sustancialmente menores para
aquellos sin CNA. El nido 5 se destacó por presentar tasas de
aprendizaje mayores al resto. Tal como se esperaba el nido 4 no presentó
las mismas tendencias que los otros nidos, mostrando una mayor tasa de
aprendizaje bajo el tratamiento sin CNA. Por lo tanto se optó por
excluir los datos correspondientes a este nido en el análisis.

\paragraph{Día:}\label{duxeda}

Para analizar la variable aleatoria Día se realizó un gráfico de
perfiles el cual muestra una gran variabilidad en las tasas de
aprendizaje promedio según el tratamiento y el día de captura. No se
observa una tendencia entre los distintos tratamientos con CNA aunque en
la amplia mayoría de los casos el tratamiento control sin compuestos
agregados presenta una menor tasa de aprendizaje. No todos los días
cuentan con datos en cada tratamiento, lo que dificulta la
interpretación general.

\begin{verbatim}
## # A tibble: 43 x 3
##    Tratamiento   Dia TasaPromedio
##    <fct>       <dbl>        <dbl>
##  1 ARG             5        0.429
##  2 ARG             6        0    
##  3 ARG            11        0.75 
##  4 ARG            12        0.75 
##  5 ARG            13        1    
##  6 ARG            14        0.333
##  7 ARG            15        0.75 
##  8 ARG            16        0.4  
##  9 ARG            17        0.5  
## 10 ARG            18        0.75 
## # i 33 more rows
\end{verbatim}

\includegraphics{Proyecto-final-Biome_files/figure-latex/Día-1.pdf}
\emph{Figura 5: Gráfico de perfiles de la tasa promedio de aprendizaje
en función del tratamiento para los distintos días evaluados.}

\subsection{Modelo}\label{modelo}

Se plantea un modelo lineal generalizado mixto de comparación de medias
con: - Tratamiento como variable de efectos fijos cualitativa con 4
niveles (CAF, ARG, CAF+ARG y SIN CNA) - Cantidad de ingestas como
variable de efectos fijos cualitativa con dos niveles (Alta y Baja) -
Nido como variable cualitativa de efectos aleatorio con tres niveles (2,
5 y 6) - Respuesta al PER como variable respuesta dicotómica (1=
extensión de probóscide solo a Lionanol, 0= no extensión al Lionanol o
extensión a Lionanol y al Nonanal) Se espera que la VR ajuste a
distribución de tipo Bernoulli.

Sea \(Y_i\) la respuesta de extensión de probóscide, donde:

\[
Y_i = 
\begin{cases} 
1 & \text{si hay extensión de probóscide} \\
0 & \text{si no hay extensión de probóscide}
\end{cases}
\]

y se modela como una variable de distribución Bernoulli:

\[
Y_i \sim \text{Bernoulli}(\pi_i)
\]

La probabilidad de extensión de probóscide, \(\pi_i\), se modela con un
enlace logit:

\[
\text{logit}(\pi_i) = \eta_i = \mu + \alpha_i + \beta_j + N_k
\]

donde:

\begin{itemize}
\tightlist
\item
  \(\mu\) es el intercepto general.
\item
  \(\alpha_i\) representa el efecto del tratamiento (con
  \(i = 1, \dots, 5\)).
\item
  \(\beta_j\) representa el efecto del nivel de condicionamiento (con
  \(j = 1\) para bajo y \(j = 2\) para alto).
\item
  \(N_k\) es el efecto del nido (con \(k = 1, 2, 3\)) que representa la
  varianza asociada al nido donde están los abejorros, introducido como
  efecto aleatorio.
\end{itemize}

\paragraph{Supuestos}\label{supuestos}

Muestra aleatoria y observaciones independientes. Normalidad para los
efectos aleatorios. Ausencia de patrones en los residuos, chequeo de
outliers.

No se encontraron evidencias para rechazar ninguno de los supuestos del
modelo, posteriormente se realizó un test de selección de modelos, para
evaluar si la inclusión de Cantidad de ingestas durante el
condicionamiento justificaba el uso de un modelo más complejo. No se
encontraron evidencias para diferenciar los modelos (Pr(Chisq)=0.5816).
Por lo tanto, se recurrió al criterio de información de Akaike (AIC), al
criterio de información bayesiano (BIC) y a la devianza para seleccionar
el modelo más adecuado. Finalmente, se optó por el modelo más simple,
dado que resultó ser el más parsimonioso (Anexo 4).

Ya seleccionado un modelo se realizó un test global en el cual no se
evidenciaron diferencias significativas entre la tasa de respuesta al
PER bajo los distintos tratamientos (Chi Sq Test = 0.09525) (Anexo 5).
Se estimó la tasa de respuesta

Tabla 5: Probabilidad media y su IC, con una confianza del 95\%, de
mostrar retención de la memoria a las 24hs para cada tratamiento.

\begin{verbatim}
##  Tratamiento  prob    SE  df asymp.LCL asymp.UCL
##  ARG         0.557 0.117 Inf     0.332     0.760
##  CAF         0.525 0.116 Inf     0.308     0.733
##  CAF+ARG     0.567 0.121 Inf     0.333     0.774
##  SIN CNA     0.293 0.112 Inf     0.125     0.545
## 
## Confidence level used: 0.95 
## Intervals are back-transformed from the logit scale
\end{verbatim}

\begin{verbatim}
##  contrast            odds.ratio    SE  df null z.ratio p.value
##  ARG / CAF                1.137 0.517 Inf    1   0.283  0.9921
##  ARG / (CAF+ARG)          0.961 0.462 Inf    1  -0.083  0.9998
##  ARG / SIN CNA            3.030 1.630 Inf    1   2.064  0.1649
##  CAF / (CAF+ARG)          0.845 0.403 Inf    1  -0.354  0.9848
##  CAF / SIN CNA            2.664 1.420 Inf    1   1.837  0.2559
##  (CAF+ARG) / SIN CNA      3.154 1.760 Inf    1   2.059  0.1669
## 
## P value adjustment: tukey method for comparing a family of 4 estimates 
## Tests are performed on the log odds ratio scale
\end{verbatim}

\includegraphics{Proyecto-final-Biome_files/figure-latex/EmmeansPlot-1.pdf}
\emph{Figura 6: Comparación de los valores medios y los IC, con una
confianza del 95\%, de la probabilidad de mostrar retención a las 24hs
para cada tratamiento.}

En el gráfico se observa la probabilidad de generar memoria asociativa
entre un olor y una recompensa a las 24 horas del entrenamiento para los
diferentes tratamientos: ARG, CAF, CAF+ARG y SIN CNA, con sus
respectivos intervalos de confianza (\alpha = 0.05). Existe una
tendencia a una mayor generación de memoria en los individuos que
consumieron cafeína y/o arginina, en particular en el grupo CAF+ARG, que
muestra la mayor probabilidad de memoria de 0,475 con un intervalo de
confianza entre 0,28 y 0,78. El tratamiento SIN CNA presenta una
probabilidad de recordar a las 24 hs de 0,209 mientras que ARG y CAF
muestran probabilidades similares de 0.475 y 0,427 respectivamente. Sin
embargo, estas diferencias no son estadísticamente significativas según
una prueba global (Chisq test = 0.1829), lo que indica que no se
encontraron evidencias para diferenciar entre sí.

Con el objetivo de verificar la hipótesis 1 se decidió entonces realizar
un contraste a priori mediante Bonferroni. Para realizarlo se
reagruparon los datos en dos grupos en base a sus tratamientos: el
primero ``Con CNA'', uniendo a los tratamientos ``Cafeína'',
``Arginina'' y ``Cafeína + Arginina'', y un segundo grupo ``Sin CNA''
que contaba con los animales que recibieron únicamente solución
azucarada durante el entrenamiento.

\begin{verbatim}
##  contrast           odds.ratio   SE  df null z.ratio p.value
##  Con CNA vs Control       2.94 1.38 Inf    1   2.296  0.0217
## 
## Tests are performed on the log odds ratio scale
\end{verbatim}

\begin{verbatim}
##  contrast           odds.ratio   SE  df asymp.LCL asymp.UCL
##  Con CNA vs Control       2.94 1.38 Inf      1.17      7.39
## 
## Confidence level used: 0.95 
## Intervals are back-transformed from the log odds ratio scale
\end{verbatim}

Tabla 5: Contraste a priori por Bonferroni entre grupo ``Con CNA'' o
``Promedio'' (que agrupa ``CAF'', ``ARG'' y ``CAF+ARG'') y grupo
``Control'' (``Sin CNA'').

Existen evidencias suficientes para concluir que los tratamientos en
abejorros con los compuestos no azucarados utilizados en este estudio
(CAF y/o ARG) tienen un efecto positivo de 0.68 a 4.36 veces mayor en la
formación de memoria a largo plazo (a las 24 horas) en comparación con
un grupo control que recibió únicamente una solución azucarada (SIN
CNA). Este resultado se obtuvo mediante una comparación a priori
utilizando la corrección de Bonferroni, con un valor de p \textless{}
0.05.

\subsection{Discusión y Conclusiones}\label{discusiuxf3n-y-conclusiones}

El consumo Cafeína y/o Arginina por parte de un abejorro Bombus
pauloensis perteneciente de un nido industrial tipo, en CABA, provoca un
incremento en promedio de aproximadamente 2,5 veces en la retención de
la memoria olfativa a largo plazo en contraste con un abejorro que
consumió únicamente solución azucarada.

Por lo tanto sería conveniente el empleo de estos compuestos en la
práctica agronómica de polinización dirigida de plantas frutales de alto
valor económico, y así reducir la utilización de especies no nativas.

Adicionalmente se recomendaría reforzar, mediante un incremento en el n
experimental, si la tendencia observada del efecto aditivo de ambos
compuestos no azucarados tendría un efecto o no en la retención de la
memoria

\section{Anexo}\label{anexo}

\subsubsection{A. 1:}\label{a.-1}

\begin{verbatim}
## # A tibble: 5 x 4
##   Tratamiento Responden Total Porcentaje
##   <fct>           <int> <int>      <dbl>
## 1 ARG                24    46       52.2
## 2 CAF                23    44       52.3
## 3 CAF+ARG            20    39       51.3
## 4 SIN CNA            10    31       32.3
## 5 SIN OLOR            0    23        0
\end{verbatim}

\emph{Anexo 1: Porcentaje de abejorros que respondieron mediante PER a
las 24hs dentro de cada tratamiento,.}

\subsubsection{A. 2:}\label{a.-2}

\begin{verbatim}
## [1] FALSE
\end{verbatim}

\begin{verbatim}
## Generalized linear mixed model fit by maximum likelihood (Laplace
##   Approximation) [glmerMod]
##  Family: binomial  ( logit )
## Formula: Recuerda ~ Tratamiento + (1 | Nido) + (1 | Dia)
##    Data: DatosMemoria
## 
##      AIC      BIC   logLik deviance df.resid 
##    207.0    225.1    -97.5    195.0      145 
## 
## Scaled residuals: 
##     Min      1Q  Median      3Q     Max 
## -1.5561 -0.8443  0.5824  0.7295  2.2117 
## 
## Random effects:
##  Groups Name        Variance Std.Dev.
##  Dia    (Intercept) 0.07592  0.2755  
##  Nido   (Intercept) 0.40925  0.6397  
## Number of obs: 151, groups:  Dia, 16; Nido, 6
## 
## Fixed effects:
##                    Estimate Std. Error z value Pr(>|z|)  
## (Intercept)         0.19222    0.48939   0.393   0.6945  
## TratamientoCAF     -0.11851    0.46358  -0.256   0.7982  
## TratamientoCAF+ARG  0.07387    0.49846   0.148   0.8822  
## TratamientoSIN CNA -1.09426    0.54372  -2.013   0.0442 *
## ---
## Signif. codes:  0 '***' 0.001 '**' 0.01 '*' 0.05 '.' 0.1 ' ' 1
## 
## Correlation of Fixed Effects:
##             (Intr) TrtCAF TCAF+A
## TratamntCAF -0.502              
## TrtmCAF+ARG -0.491  0.485       
## TrtmnSINCNA -0.431  0.436  0.411
\end{verbatim}

\emph{A.2: Al ajustar el modelo con las variables aleatorias Día y Nido,
se observa el mensaje boundary(Singular), que indica problemas de
singularidad en el modelo. Esto quiere decir que alguna de las variables
aleatorias no aporta suficiente variabilidad y, por lo tanto, no
contribuye de manera significativa al modelo. Esto fue confirmado al
ejecutar isSingular(MMixtoNidoDia) devolviendonos TRUE. Además, al
correr el summary del modelo (Anexo 2), observamos que la varianza de la
variable aleatoria Día es muy baja, especialmente en comparación con la
varianza de la variable Nido.}

\subsubsection{A.3}\label{a.3}

\includegraphics{Proyecto-final-Biome_files/figure-latex/Supuestos-1.pdf}

\emph{A3.1: Prueba de ajuste a una distribución uniforme usando un
QQplot de residuos re-escalados (izquierda) y evaluación de desviación
de la uniformidad y homogeneidad de varianza (derecha) utilizando el
paquete DHARMa.}

En el panel izquierdo, se presenta un gráfico QQ de los residuos, donde
se observa la distribución de los residuos simulados comparada con la
distribución teórica esperada (línea roja). Los resultados de varias
pruebas de ajuste también se incluyen en este panel: el test de
Kolmogorov-Smirnov (KS test) muestra un valor de p = 0.06095, indicando
que no hay desviaciones significativas de la normalidad; el test de
dispersión tiene un valor de p = 0.376, lo que sugiere que no hay
desviación significativa en la dispersión; y el test de outliers (valor
de p = 1) indica que no se encontraron valores atípicos significativos
en los residuos.

En el panel derecho, se muestra un gráfico (boxplot) que evalúa la
homogeneidad de los residuos a través de los grupos predictores. El test
de Levene, indicado en el gráfico, muestra que no hay diferencias
significativas en la varianza de los residuos entre grupos, sugiriendo
homogeneidad de varianza. En general, los resultados de ambas gráficas
indican que los residuos cumplen con los supuestos de normalidad,
homogeneidad y ausencia de outliers, lo que respalda la validez del
modelo ajustado.

\subsubsection{A.4: Criterio de selección de
modelo}\label{a.4-criterio-de-selecciuxf3n-de-modelo}

\begin{verbatim}
## Data: DatosMemoria
## Models:
## MMixtoNido: Recuerda ~ Tratamiento + (1 | Nido)
## MMixto2: Recuerda ~ Tratamiento + Condicionamiento2 + (1 | Nido)
##            npar    AIC    BIC  logLik deviance  Chisq Df Pr(>Chisq)
## MMixtoNido    5 205.08 220.16 -97.539   195.08                     
## MMixto2       6 206.12 224.23 -97.063   194.12 0.9525  1     0.3291
\end{verbatim}

\emph{A.4: AIC contrastando modelo mixto con o sin ``Ingestas'' como
variable fija.}

\subsubsection{A.5: Criterio de selección de
modelo}\label{a.5-criterio-de-selecciuxf3n-de-modelo}

\begin{verbatim}
## Generalized linear mixed model fit by maximum likelihood (Laplace
##   Approximation) [glmerMod]
##  Family: binomial  ( logit )
## Formula: Recuerda ~ Tratamiento + (1 | Nido)
##    Data: DatosMemoria
## 
##      AIC      BIC   logLik deviance df.resid 
##    205.1    220.2    -97.5    195.1      146 
## 
## Scaled residuals: 
##     Min      1Q  Median      3Q     Max 
## -1.5441 -0.8695  0.6476  0.7047  2.3293 
## 
## Random effects:
##  Groups Name        Variance Std.Dev.
##  Nido   (Intercept) 0.4217   0.6494  
## Number of obs: 151, groups:  Nido, 6
## 
## Fixed effects:
##                    Estimate Std. Error z value Pr(>|z|)  
## (Intercept)         0.22836    0.47214   0.484    0.629  
## TratamientoCAF     -0.12879    0.45467  -0.283    0.777  
## TratamientoCAF+ARG  0.04008    0.48077   0.083    0.934  
## TratamientoSIN CNA -1.10848    0.53694  -2.064    0.039 *
## ---
## Signif. codes:  0 '***' 0.001 '**' 0.01 '*' 0.05 '.' 0.1 ' ' 1
## 
## Correlation of Fixed Effects:
##             (Intr) TrtCAF TCAF+A
## TratamntCAF -0.500              
## TrtmCAF+ARG -0.468  0.481       
## TrtmnSINCNA -0.430  0.431  0.403
\end{verbatim}

Anexo 5.1: Summary del modelo mixto seleccionado.

\begin{verbatim}
## Analysis of Deviance Table (Type II Wald chisquare tests)
## 
## Response: Recuerda
##              Chisq Df Pr(>Chisq)
## Tratamiento 5.3682  3     0.1467
\end{verbatim}

Anexo 5.2: Test Global mediante Chi Sq. del modelo seleccionado. No se
obtiene una diferencia significativa (Chi Sq Test = 0.09525) .

\end{document}
